% cv.tex
% vim:set ft=tex spell:

%\documentclass[10pt,a4paper]{article}
\documentclass{article}
\usepackage[letterpaper,margin=0.75in]{geometry}
\usepackage[utf8]{inputenc}
\usepackage{mdwlist}
\usepackage[T1]{fontenc}
\usepackage{textcomp}
\usepackage{tgpagella}
%use headers and footers package
\usepackage{fancyhdr}
%show headers and footers
\pagestyle{fancy}
%change the starting offset for the page numbers
\setcounter{page}{1}
%define a centred header
%\chead{Gregory Cartwright -- \emph{Curriculum Vitae}}
%define a centred footer
\cfoot{Gregory Cartwright}
\setlength{\tabcolsep}{0em}

% indentsection style, used for sections that aren't already in lists
% that need indentation to the level of all text in the document
\newenvironment{indentsection}[1]%
{\begin{list}{}%
	{\setlength{\leftmargin}{#1}}%
	\item[]%
}
{\end{list}}

% opposite of above; bump a section back toward the left margin
\newenvironment{unindentsection}[1]%
{\begin{list}{}%
	{\setlength{\leftmargin}{-0.5#1}}%
	\item[]%
}
{\end{list}}

% format two pieces of text, one left aligned and one right aligned
\newcommand{\headerrow}[2]
{\begin{tabular*}{\linewidth}{l@{\extracolsep{\fill}}r}
	#1 &
	#2 \\
\end{tabular*}}

% make "C++" look pretty when used in text by touching up the plus signs
%\newcommand{\CPP}
%{C\nolinebreak[4]\hspace{-.05em}\raisebox{.22ex}{\footnotesize\bf ++}}

% and the actual content starts here
\begin{document}

\begin{center}
{\LARGE \textbf{Gregory Cartwright}}

118 Chapel Street\ \ \textbullet
\ \ Ibstock\ \ \textbullet
\ \ Leicestershire\ \ \textbullet
\ \ LE67 6HG
\\
07977 22 88 38\ \ \textbullet
\ \ GregCartwright@tuta.com
\end{center}

\hrule
\vspace{-0.4em}
\subsection*{Employment Experience}

\begin{itemize}
	\parskip=0.1em

	\item
	\headerrow
		{\textbf{Nottingham CityCare Partnership}}
		{\textbf{Nottingham}}
	\\
	\headerrow
		{\emph{Lead Advanced Clinical Practitioner, Urgent Community Response \& Reablement}}
		{\emph{July 2023 -- present}}
	\begin{itemize*}
		\item Leading Advanced Practice within the UCR team and acting as a senior clinical resource for the whole team.
		\item Working closely with the leadership team to continuously evaluate and develop the service, maximising patient outcomes and ensuring safety.
		\item Representing the team and the organisation within multi-agency fora to ensure that UCR develops along with other parts of the System to meet the needs of the local population.
		\item Autonomous assessment and management of acutely deteriorating people in their place of residence, balancing risk to ensure optimal outcomes.
	\end{itemize*}

	\item
	\headerrow
		{\textbf{South Warwickshire University NHS Foundation Trust}}
		{\textbf{Nuneaton}}
	\\
	\headerrow
		{\emph{Advanced Clinical Practitioner, Urgent Community Response}}
		{\emph{May 2022 -- July 2023}}
	\begin{itemize*}
		\item Working autonomously in domicilliary settings to assess and treat detiorating people, aiming to avoid hospital admission if possible.
		\item Supporting junior members of the team as a senior clinical resource.
		\item Undertaking the role of Trust lead for non-medcial prescribing.
	\end{itemize*}

	\item
	\headerrow
		{\textbf{Nottingham University Hospitals NHS Trust}}
		{\textbf{Nottingham}}
	\\
	\headerrow
		{\emph{Advanced Clinical Practitioner, Acute Medicine}}
		{\emph{December 2019 -- May 2022}}
	\begin{itemize*}
		\item Assessment and initial mangement of acutely ill adult patients with undifferentiated problems.
		\item Ordering and interpretation of investigations to assist in the clinical reasoning process.
		\item Triage and follow-up of patients referred to Acute Medicine from the Emergency Department.
		\item Leading on and participating in quality improvement projects within the department.
	\end{itemize*}

	\item
	\headerrow
		{\textbf{Leicestershire Partnership NHS Trust}}
		{\textbf{Leicestershire}}
	\\
	\headerrow
		{\emph{Advanced Nurse Practitioner}}
		{\emph{January 2015 -- present}}
	\begin{itemize*}
		\item Working autonomously in a community hospital ward to provide the 
			clinical management of the inpatients.
		This is done under the supervision of the acute consultant who visits 
			the ward twice a week.

		\item Providing leadership within the MDT to maximise patient outcomes 
			and minimise delays to their pathways, 
			with the aim of achieving optimal patient experience. 
		
	\end{itemize*}

	\item
	\headerrow
		{\textbf{University Hospitals of Leicester NHS Trust}}
		{\textbf{Leicester}}
	\\
	\headerrow
		{\emph{Critical Care Outreach Nurse}}
		{\emph{April 2009 -- December 2014}}
	\begin{itemize*}

	\item Providing support for critically ill patients in ward environments.
	\item Working with ward-based medical and nursing staff to optimise 
	outcomes for these patients.
	\item Assessing patients using an ABCDE structure to identify the 
	current issues affecting their current health.
	\item Liaising with the patients and the teams on the wards to fomulate
	plans to address these issues.
	\item When necessary, working closely with the team on the Critical
	Care Unit to arrange patient review or admission to the unit.
	\item Providing education for clinical staff including nurses, 
	junior doctors, and students.
	\end{itemize*}

%	\item
	\headerrow
		{\emph{Staff Nurse, Respiratory Medicine}}
		{\emph{September 2008 -- April 2004}}
	\begin{itemize*}
		\item Providing evidence based care for in-patients on an acute 
		respiratory ward.
		\item Working as part of a multidisciplinary team.
		\item Leading the ward on a daily basis.
	\end{itemize*}
%	\item
	\headerrow
		{\emph{Staff Nurse, Intensive Care Unit}}
		{\emph{September 2005 -- September 2008}}
	\begin{itemize*}
		\item Providing all care for critically ill patients in a busy 
		intensive care unit.
		\item Working as part of a multidisciplinary team.
		\item Contributing to education programmes for junior staff.
	\end{itemize*}
		

\end{itemize}


\hrule
\vspace{-0.4em}
\subsection*{Education}

\begin{itemize}
	\parskip=0.1em

	\item 
	\headerrow
		{\textbf{London South Bank University}}
		{\textbf{London}}
	\\
	\headerrow
		{\emph{MSc Advanced Nurse Practitioner [RCN Accredited]}}
		{\emph{September 2015 -- July 2018}}
	This included clinical modules, plus:
	\begin{itemize*}
		\item RCN-7-013 "Leadership and Service Development for ANPs"
		\item Dissertation "Does level of frailty influence advance care planning
			for community hospital inpatients?"
	\end{itemize*}
%	\item
	\headerrow
		{\emph{PgCert Non-Medical Prescribing}}
		{\emph{January 2015 -- August 2015}}
	\item 
	\headerrow
		{\textbf{De Montfort University}}
		{\textbf{Leicester}}
	\\
	\headerrow
		{\emph{PgCert Accountability and Consultation Skills with Merit}}
		{\emph{September 2013 -- November 2014}}
%	\item
	\\
	\headerrow
		{\emph{Mentorship in Clinical Healthcare 15 credit degree level module}}
		{\emph{January 2008 -- July 2008}}
	\\
	\headerrow
		{\emph{BSc (Hons) Second Class (Upper Division) Nursing (Adult Branch)}}
		{\emph{September 2002 -- October 2005}}
	\item 
	\headerrow
		{\textbf{University Of Warwick}}
		{\textbf{Coventry}}
	\\
	\headerrow
		{\emph{BSc (Hons) in Mathematics Class Two: Division Two}}
		{\emph{September 1993 -- July 1996}}
	
\end{itemize}


\hrule
\vspace{-0.4em}
\subsection*{Short Vocational Courses}

\begin{itemize}
	\item
	\headerrow
		{\textbf{Two Day Interactive Communication Skills Course}}
		{\textbf{3\textsuperscript{rd} and 4\textsuperscript{th} August 2015}}
	\\
	\headerrow
		{\emph{LOROS Hospice}}
		{\emph{Leicester}}

	\item
	\headerrow
		{\textbf{Advanced Life Support Course}}
		{\textbf{14\textsuperscript{th} and 15\textsuperscript{th} May 2015}}
	\\
	\headerrow
		{\emph{University Hospitals of Leicester}}
		{\emph{Leicester}}
	
	\item
	\headerrow
		{\textbf{Advnaced Life Support Course}}
		{\textbf{10\textsuperscript{th} March 2021}}
	\\
	\headerrow
		{\emph{Nottingham University Hospitals}}
		{\emph{Nottingham}}

\end{itemize}
\end{document}
